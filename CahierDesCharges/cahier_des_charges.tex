% Crazy Corp
% Crazy Delivery
% 2022
% Cahier des charges

% Librairies
\documentclass[11pt,a4paper]{article}
\usepackage[utf8]{inputenc}
\usepackage[french]{babel}
\usepackage[T1]{fontenc}
\usepackage{graphicx}
\usepackage{mathtools}
\usepackage{url}

% Commandes
\setlength\parindent{15pt}
\newcommand{\cd}{\textit{Crazy Delivery}}
\newcommand{\AI}{Intelligence Artificielle}

% Début du document
\begin{document}

\title{Cahier des Charges \\ {\large Crazy Delvery}}
\author{Crazy Corp}
\date{Janvier 2022 - S2}
\maketitle

% Introduction de l'image du jeu
\begin{center}
  \includegraphics[width=0.5\textwidth]{}
\end{center}

\tableofcontents
\clearpage

% Introduction (1)
\section{Introduction}
  
\clearpage

% Origine et Nature (2)
\section{Origine et Nature du projet}
À la base nous voulions nous lancer sur un Mario kart avec des caddies, mais nous avons décidé de trouver une autre type de jeu après avoir découvert que cela avait déjà été produit par une promotion antérieure. Nous tenions absolument à ce que ce jeu n'ait pas encore été crée, du moins pas dans les précédents projets. À court d'idées, nous avons chacun fait un sondage auprès de nos proches et l'idée la plus originale qui a retenue notre attention était un jeu de livraison de fast-food. Nous nous sommmes alors inspiré d'\textit{Uber Eat}.Le nom Crazy Delivery nous est alors venu à l'esprit, avec "Crazy" qui veut dire fou en anglais, en référence au jeu Crazy Taxi (cf. État de l’art) mais avec l'anglais "Delivery" associé aux livraisons.
La carte de jeu sera donc une ville animée d'un traffic routier intelligent (mise en place d'une \AI).



% Objet d'etude (3)
\section{Objet de l'étude}
 Dans notre future carrière d'Ingenieur, nous devrons apporter des solutions innovantes à nos clients.
  Pour cela, il nous est nécessaire d'apprendre à organiser les différentes étapes de nos projets, à communiquer avec les membres de notre équipe ainsi qu'à prendre en compte les requêtes de nos clients (ici l’EPITA). Pour ce faire, il n’y a rien de plus formateur que "mettre les mains dans le cambouis". Nous serons confrontés à des difficultés, des retards, des problèmes logiciels, des nuits blanches, etc. Mais c’est à travers ce premier projet que nous construirons nos bases logistiques et expérimentales pour notre future carrière, et ce, dès notre première année d’étude. Voici à présent un message de chacun de nos membres sur notre intérêt personnel face à ce projet, et plus globalement en quoi il nous sera façonneur, socialement comme techniquement. \\
  
  \textit{Thybalt} \\
  Cela fait environ un an que j'ai hâte de me lancer dans ce projet. Effectivement, ce fut un point décisif lors du choix de mon école l’année passée. Nous allons pouvoir appliquer notre savoir accumulé depuis le début de l’année en programmation pour construire un jeu complet et agréable (je l’espère) pour les utilisateurs. Cela va non seulement consolider mes connaissances en C\textbf{\#} et apprendre à utiliser des logiciels comme Unity, ou encore mieux connaitre le fonctionnement de GitHub, mais aussi à découvrir la modélisation graphique 3 dimensions et également le système de multijoueur. Ce sera également l’occasion d’améliorer de manière conséquente mes capacités d’organisation qui sont déjà sollicitées au quotidien en prépa. Enfin, ce projet me donnera une première idée des projets que nous pourrions être amené à concevoir en entreprise et ne pourra que confirmer mon choix d’orientation en école d’Ingenieur. Ce projet jeu vidéo contribuera à mon épanouissement dans multiples domaines et je m’en rappellerai pendant toute ma carrière. \\
  \\
  \textit{Antoine} \\
  Aillant déjà participé à de multiples travaux de groupes je savais que ce projet ne serais pas une mince à faire. Cependant il ne peut être que enrichissant et divertissant. Le travaille en groupe/société est quelque chose qui s’apprend à travers les années, il est donc nécessaire de réaliser des petits projets en groupe pour pouvoir se préparer à ce que nous réserve l’avenir. Cela nous permet de développer notre savoir vivre. De plus nous allons toucher à plusieurs domaines, IA, graphisme, etc, d’où l’importance que je porte à ce projet, car il promet d’être une flux de savoir conséquent.

% Etat de l'art (4)
\section{État de l'art}
  L’un des premiers jeu de course chronométré est Crazy Taxi (1999). Dans ce jeu vous pouvez incarner un taxi qui doit faire des courses en un temps record. Cependant, notre jeu a la particularité en plus d'avoir le monde des « \textit{fast food }» dans le but de rendre l'experience plus dynamique et originale. On peut également citer le mode course contre la montre de Mario Kart (1992) mais qui lui reste sur un circuit fermé. En 2006, Roblox fait son apparition, il tend  à être l’un des jeux les plus joué de sa génération. Parmi ses nombreux modes, on peut retrouver un mode livreur de pizza ce qui est étroitement relié à notre livreur de fast food. Notre jeu est donc bien un jeu de livraison chronométrées, mais dans un OPEN WORLD, ce qui apporte une expérience de jeu plus attrayante pour le joueur. De plus, la compétition contre d’autres joueurs ajoute un aspect stimulant qui poussera les joueurs à donner leur maximum pour remporter le plus de points.

% Découpage du Projet (5)
\section{Découpage du projet}
\subsection{Moyens techniques}
    \subsubsection{Moteur de jeu}
    Afin de réaliser notre jeu dans les meilleurs conditions nous avons choisit d’utiliser Unity 3D pour sa conception et de programmer en C\textbf{#} avec Rider ou VSCode. Unity 3D propose un large panel de fonctionnalités et ainsi semble être suffisant pour créer notre jeu. Cette partie regroupe alors la gestion des mécaniques de jeu, le déplacement des personnages, ainsi que l'implémentation des différentes règles propres au focntionnement de notre jeu vidéo.
      
    \subsubsection{Graphisme}
    Cette section va regrouper tout le travail graphique lié au jeu. Tout d'abord, il y a la conception de la carte de jeu. En effet, grâce à un pack d’assets Unity, nous allons devoir construire notre ville de A à Z, en se basant sur un plan de la ville fait préalablement au croquis. Nous allons également devoir faire de la modélisation graphique en 3 dimensions avec le logiciel Blender pour modifier certaines structures de bâtiments (pour créer un restaurant Speed Burger si on obtient les droits par exemple). Il y aura également un travail de retouche sur la 2D avec Gimp, pour designer l’écran d’accueil ou les différents menus du jeu par exemple
      
    \subsubsection{Multijoueur et \AI}
      
    \subsubsection{Communication}
    --> Bien parler de Speed Burger
    \subsubsection{Répartition des tâches}
      --> Tableau responsable + suppléant 

  \subsection{Planning}
    --> Tableau à faire

\section{Organisation et méthodologie}
L'organisation est primordiale pour aboutir à temps à nos objectifs. Nous avons donc dû mettre en place des solutions pour y parvenir dans les semaines à venir. \\
Tout d'abord pour la répartition des tâches, nous partageons un fichier \textit{Excel} où l'on peut retrouver un onglet Actions préparatoires avec la ou les personnes concernée(s), le statut (À commencer, En cours, Terminé), ainsi que la date limite pour la compléter. Il y a également un autre onglet Chronogramme qui permet d’obtenir un vison temporel des tâches à effectuer pendant les différentes périodes du projet.  Enfin, il y a un dernier onglet nommé « Synthèse de Projet » avec des diagrammes pour chaque période (entre deux soutenances) avec le statut des taches affiché par différentes couleurs, nous permettant une vue d’ensemble sur l’avancée du projet. \\ \\
Ensuite, bien qu’il y ait des responsables pour les différentes sections du projet, l’objectif est que tout le monde s’entraide dans les différentes tâches et qu’une personne ne s’occupe pas de toute sa partie. Pour cela, nous avons créer un espace \textit{GitHub} qui nous permet à travers différents repositories de partager les fichiers du jeu à tout les membres pour que chacun puisse faire avancer le projet à sa manière. Nous utilisons par ailleurs le logiciel \textit{Sourcetree} pour gérer correctement la sauvegarde du contenu. De plus, pour communiquer, s’échanger des liens et programmer des réunions, nous utilisons notre propre serveur sur le réseau \textit{Discord}.\\ \\
Enfin, pour assurer le bon avancement ainsi qu’un suivi régulier du projet, nous organisons des réunions toutes les semaines une réunion pendant laquelle nous faisons un briefing sur l’avancée des tâches en cours, mais aussi la mise au point de nouveaux objectifs et c’est l’occasion de montrer au groupe et de corriger le travail effectué personnellement au préalable.\\

\clearpage
% Conclusion (5)
\section{Conclusion}

\begin{bibliography}{9}
  \bibitem{instagram}
    \url{https://www.instagram.com/crazy__corp/}

\end{bibliography}

\textcopyright\, 2021-2022 \cd
\end{document}